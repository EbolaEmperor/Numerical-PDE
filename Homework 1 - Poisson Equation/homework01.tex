\documentclass[twoside,a4paper]{article}
\usepackage{geometry}
\geometry{margin=1.5cm, vmargin={0pt,1cm}}
\setlength{\topmargin}{-1cm}
\setlength{\paperheight}{29.7cm}
\setlength{\textheight}{25.3cm}

% useful packages.
\usepackage{amsfonts}
\usepackage{amsmath}
\usepackage{amssymb}
\usepackage{amsthm}
\usepackage{enumerate}
\usepackage{graphicx}
\usepackage{multicol}
\usepackage{fancyhdr}
\usepackage{layout}

% some common command
\newcommand{\dif}{\mathrm{d}}
\newcommand{\avg}[1]{\left\langle #1 \right\rangle}
\newcommand{\difFrac}[2]{\frac{\dif #1}{\dif #2}}
\newcommand{\pdfFrac}[2]{\frac{\partial #1}{\partial #2}}
\newcommand{\OFL}{\mathrm{OFL}}
\newcommand{\UFL}{\mathrm{UFL}}
\newcommand{\fl}{\mathrm{fl}}
\newcommand{\op}{\odot}
\newcommand{\Eabs}{E_{\mathrm{abs}}}
\newcommand{\Erel}{E_{\mathrm{rel}}}

\begin{document}

\pagestyle{fancy}
\fancyhead{}
\lhead{Wenchong Huang (3200100006)}
\chead{Numerical PDE homework \#1}
\rhead{Mar. 19th, 2023}


\section*{I. Exercise 7.14 (Check the different norms.)}

\begin{equation*}
	||\mathbf{g}||_\infty = \max_{1\leq i\leq N} |g_i| = \max\{O(h), O(h^2)\} = O(h).
\end{equation*}

Note that $hN$ is $O(1)$, we have:
\begin{equation*}
	||\mathbf{g}||_1=h\sum_{i=2}^{N-1} O(h^2) + 2hO(h) = O(h^2) + O(h^2) = O(h^2)
\end{equation*}

\begin{equation*}
	||\mathbf{g}||_2=\left(h\sum_{i=2}^{N-1} O(h^4) + 2hO(h^2)\right)^\frac{1}{2}=\left(O(h^4) + O(h^3)\right)^\frac{1}{2} = O\left(h^\frac{3}{2}\right)
\end{equation*}

\section*{II. Exercise 7.35 (Show the elements in the first column of $A_E^{-1}$ are $O(1)$.)}

\begin{equation*}
	A_E=\frac{1}{h^2}\begin{bmatrix}
		-h & h  &   & \\
		1  & -2 & 1 &  \\
		   & 1  & -2& 1  \\
		   &    & \ddots & \ddots & \ddots  \\
		   &    &        &  1 & -2 & 1 \\
		   &    &        &    &  1 & -2 & 1\\
		   &    &        &    &    &  0 & h^2
	\end{bmatrix}.
\end{equation*}

Suppose $v$ is the first column of $A_E^{-1}$, which is uniquely determined by
\begin{equation*}
	A_Ev=e_1
\end{equation*}

where $e_1=[1,0,0,...,0]^T$. Solve the equation above, we got
\begin{equation*}
	v=-h\begin{bmatrix}m+1 \\ m \\ \vdots \\ 1 \\ 0\end{bmatrix}
\end{equation*}

Note that $(m+1)h=1$, hence the elements in $v$ are $O(1)$.

\section*{III. Exercise 7.40 (Estimate LTE of the FD method in two dimensions.)}

\;\;\;\; For simplicity, we will write $\frac{\partial^k u}{\partial x^k},\;\frac{\partial^k u}{\partial y^k}$ instead of $\frac{\partial^k u}{\partial x^k}(x_i,y_j),\;\frac{\partial^k u}{\partial y^k}(x_i,y_j)$.

By using Taylor expansion, we derive that
\begin{equation*}
	U_{i-1,j} = U_{ij} - h\frac{\partial u}{\partial x} + \frac{h^2}{2}\frac{\partial^2 u}{\partial x^2}
	-\frac{h^3}{6}\frac{\partial^3 u}{\partial x^3} + \frac{h^4}{24}\frac{\partial^4 u}{\partial x^4}\
	-\frac{h^5}{120}\frac{\partial^5 u}{\partial x^5} + O(h^6),\\
\end{equation*}

\begin{equation*}
	U_{i+1,j} = U_{ij} + h\frac{\partial u}{\partial x} + \frac{h^2}{2}\frac{\partial^2 u}{\partial x^2}
	+\frac{h^3}{6}\frac{\partial^3 u}{\partial x^3} + \frac{h^4}{24}\frac{\partial^4 u}{\partial x^4}\
	+\frac{h^5}{120}\frac{\partial^5 u}{\partial x^5} + O(h^6).
\end{equation*}

Sum them up, we got
\begin{equation*}
	U_{i-1,j}-2U_{ij}+U_{i+1,j} = h^2\frac{\partial^2 u}{\partial x^2} + \frac{h^4}{12}\frac{\partial^4 u}{\partial x^4} + O(h^6).
\end{equation*}

Similarly we have
\begin{equation*}
	U_{i,j-1}-2U_{ij}+U_{i,j+1} = h^2\frac{\partial^2 u}{\partial y^2} + \frac{h^4}{12}\frac{\partial^4 u}{\partial y^4} + O(h^6).
\end{equation*}

And hence that
\begin{align*}
	\tau_{i,j} &= -D^2u(x_i,y_j) - (-\Delta u(x_i,y_j)) \\
	&= -\frac{U_{i-1,j}-2U_{ij}+U_{i+1,j}}{h^2}-\frac{U_{i,j-1}-2U_{ij}+U_{i,j+1}}{h^2}+\frac{\partial^2 u}{\partial x^2}+\frac{\partial^2 u}{\partial y^2}\\
	&= -\frac{h^2\frac{\partial^2 u}{\partial x^2} + \frac{h^4}{12}\frac{\partial^4 u}{\partial x^4} + O(h^6)}{h^2} - \frac{h^2\frac{\partial^2 u}{\partial y^2} + \frac{h^4}{12}\frac{\partial^4 u}{\partial y^4} + O(h^6).}{h^2} + \frac{\partial^2 u}{\partial x^2}+\frac{\partial^2 u}{\partial y^2}\\
	&= -\frac{\partial^2 u}{\partial x^2}+\frac{h^2}{12}\frac{\partial^4 u}{\partial x^4}+O(h^4)-\frac{\partial^2 u}{\partial y^2}+\frac{h^2}{12}\frac{\partial^4 u}{\partial y^4}+O(h^4)+ \frac{\partial^2 u}{\partial x^2}+\frac{\partial^2 u}{\partial y^2}\\
	&= -\frac{h^2}{12}\left.\left(\frac{\partial^4 u}{\partial x^4}+\frac{\partial^4 u}{\partial y^4}\right)\right|_{(x_i,y_j)}+O(h^4)
\end{align*}

\section*{IV. Exercise 7.60 (Estimate LTE of irregular and regular points.)}

\;\;\;\; Since $\overline{\Omega}=[0,1]^2$ is compact, the value
\begin{equation*}
	M=\max_{\overline{\Omega}} \left|\frac{\partial^4 u}{\partial x^4}+\frac{\partial^4 u}{\partial y^4}\right|
\end{equation*}

exists. Hence by Exercise 7.40, for the regular point $Q$ we have:
\begin{equation*}
	|\tau_Q|=\left|-\frac{h^2}{12}\left.\left(\frac{\partial^4 u}{\partial x^4}+\frac{\partial^4 u}{\partial y^4}\right)\right|_Q+O(h^4)\right| \leq \frac{M}{12}h^2+O(h^4).
\end{equation*}

Hence $\tau_Q$ is $O(h^2)$. Now for the irregular point $P$ we have
\begin{equation*}
	L_hU_P=\frac{(1+\theta)U_P-U_A-\theta U_W}{\frac{1}{2}\theta(1+\theta)h^2}+\frac{(1+\alpha)U_P-U_B-\alpha U_S}{\frac{1}{2}\alpha(1+\alpha)h^2},
\end{equation*}

where
\begin{equation*}
U_A=U_P+\theta h\frac{\partial u}{\partial x}+\frac{1}{2}\theta^2h^2\frac{\partial^2 u}{\partial^2 x} + O(h^3),
\end{equation*}

\begin{equation*}
U_W=U_P-h\frac{\partial u}{\partial x}+\frac{1}{2}h^2\frac{\partial^2 u}{\partial^2 x} + O(h^3).
\end{equation*}

Then we have
\begin{equation*}
	(1+\theta)U_P-U_A-\theta U_W = -\frac{1}{2}\theta(\theta+1)h^2\frac{\partial^2 u}{\partial x^2} + O(h^3).
\end{equation*}

Similarly we have
\begin{equation*}
	(1+\alpha)U_P-U_B-\alpha U_S = -\frac{1}{2}\alpha(\alpha+1)h^2\frac{\partial^2 u}{\partial x^2} + O(h^3).
\end{equation*}

Hence
\begin{equation*}
	L_hU_P=-\frac{\partial^2 u}{\partial x^2}-\frac{\partial^2 u}{\partial y^2}+O(h).
\end{equation*}

It follows that

\begin{equation*}
	\tau_P = L_hU_P - (-\Delta u(P)) = O(h)
\end{equation*}


\section*{V. Exercise 7.62 (Prove Theorem 7.61.)}

\;\;\;\; Define
\begin{equation*}
	\psi:\mathbf{X}\to\mathbb{R},\;\psi_P=E_P+T_\text{max}\phi_P,
\end{equation*}

where $T_\text{max}=\max\left\{\frac{T_1}{C_1},\frac{T_2}{C_2}\right\}$. Now $\forall P\in\mathbf{X}_1$, we have
\begin{equation*}
	L_h\psi_P=L_h(E_P+T_\text{max})=T_P+T_mL_h\phi_P\leq T_P-\frac{T_1}{C_1}C_1<0.
\end{equation*}

Similarly, $\forall P\in\mathbf{X}_2$
\begin{equation*}
	L_h\psi_P=L_h(E_P+T_\text{max})=T_P+T_mL_h\phi_P\leq T_P-\frac{T_2}{C_2}C_2<0.
\end{equation*}

Hence $L_h\psi_P<0$ hold for all $P\in\mathbf{X}$. Furthermore, we have $\max_{P\in\mathbb{X}} \psi_P\geq 0$ because $\phi_P\geq 0$ and
\begin{equation*}
	\forall Q\in\mathbf{X}_{\partial \Omega},\quad E_Q=0.
\end{equation*}

Then by Lemma 7.56 we have
\begin{equation*}
	E_P\leq \max_{P\in\mathbf{X}_\Omega}(E_P+T_\text{max}\phi_P)\leq \max_{Q\in\mathbf{X}_{\partial\Omega}}(E_P+T_\text{max}\phi_Q)=T_\text{max}\max_{Q\in\mathbf{X}_{\partial\Omega}}(\phi_Q).
\end{equation*}

Repeat the above arguments with $\psi_P=-E_P+T_\text{max}\phi_P$ and we have
\begin{equation*}
	-E_P\leq T_\text{max} \max_{Q\in\mathbf{X}_{\partial\Omega}}(\phi_Q).
\end{equation*}

\end{document}

%%% Local Variables: 
%%% mode: latex
%%% TeX-master: t
%%% End: 
